% !BIB TS-program = biber
% !BIB program = biber

\documentclass{article}
\usepackage[utf8]{inputenc}
\usepackage[top= 2cm, bottom=2cm, left=2cm, right=2cm]{geometry}
\usepackage{amsmath, amsfonts, graphicx}
\usepackage{url}

\usepackage[backend=biber, sorting=none]{biblatex}
\addbibresource{references.bib}

\begin{document}

\title{CS291D Final Report: a Basic Zcash Implementation}
\author{Gwyneth Allwright, Karl Wang, Dewei Zeng}

\maketitle

\section*{Abstract}
In this project, we attempt a basic implementation of Zerocash \cite{zerocash} in Python. Zerocash is a ledger-based digital currency that makes use of zero-knowledge Succinct Non-Interactive Arguments of Knowledge (zk-SNARKs) to provide stronger privacy guarantees than currencies such as Bitcoin \cite{bitcoin} and Zerocoin \cite{zerocoin}. This functionality is provided through a decentralized anonymous payment (DAP) scheme that hides a transaction's origin, destination and amount. We follow \cite{zerocash} to implement the following core functions: \texttt{Setup}, \texttt{CreateAddress}, \texttt{Receive}, \texttt{Mint}, \texttt{VerifyTransaction} and \texttt{Pour}, which form the foundations of Zerocash.

\tableofcontents

\newpage

\section{Introduction}
\section{Problem Definition}
\section{Solution}
\section{Related Work}
\section{Evaluation}

\printbibliography

\end{document}